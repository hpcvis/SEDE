% This LaTex template is for manuscripts to be included in
% ISCA conference proceedings.
%
\documentclass[letterpaper,twocolumn]{article}
%
\usepackage[raggedright]{titlesec}
%
% include packages as needed, such as
%\usepackage{amsmath}
%\usepackage{amsthm}
%\usepackage{graphicx}
%\usepackage[ruled,vlined]{algorithm2e}
%\usepackage{array}
%\usepackage{multirow}
%\usepackage{url}
%\usepackage{subfigure}
%etc.

%no page numbers
\pagestyle{empty}

\setlength{\textwidth}{7.0in}
\setlength{\textheight}{9.125in}
\setlength{\columnsep}{0.375in}
\setlength{\topmargin}{-0.8in}
\setlength{\oddsidemargin}{-0.25in}
\setlength{\evensidemargin}{-0.25in}
\setlength{\parindent}{0.125in}
\setlength{\parskip}{1mm}

\renewcommand\labelenumi{(\theenumi)}

\date{}

\titlespacing{\section}{0pt}{*2.0}{*2.0}
\titlespacing{\subsection}{0pt}{*1.8}{*1.8}

\renewcommand{\topfraction}{0.9}
\renewcommand{\textfraction}{0.1}

\hyphenpenalty=5000
\tolerance=1000

\begin{document}

% Change to your title
\title{\Large\textbf{Latex Template for ISCA Conference Papers}\normalsize}

%Chane authors, institutions and emails
\author {John Doe and Jane Alan\\
Department of Rocket Science, Some University\\
City, State/Province, Zip/Post Code, Country\\
(jdoe, jalen)@mailserver.edu
}
\maketitle 

\thispagestyle{empty}

\begin{center}
\large\textbf{Abstract}
\end{center}

\vspace{2mm}
Place the word \textbf{Abstract} (12-point bond Times Roman font) at the center of the column. Leave a space line below \textbf{Abstract} and then start the text (10-point Times Roman font) of the abstract of the paper. Limit the abstract to 200 words.

\medskip
\noindent
\textbf{keywords:} This part is optional. The word \textbf{keywords} is flushed to the left (no indentation) with 10-point bold font. Include up to 6 keywords. %One space line below the abstract.

\section{Introduction} 

%The sections are just examples. You should make your own sections and sub-sections.

\section{Related Work}

\section{Methodology}

\section{Experimental Results}

\section{Conclusion}

%You may put all reference items in a separate file, say myRef.bib, in bibTex format.
%
%Here is an example of a myRef.bib file that contains three reference items:
%
%@article{Liben-Nowell-2007,
%author    = {David Liben-Nowell and Jon Kleinberg},
%title     = {The Link Prediction Problem for Social Networks},
%journal   = {Journal of the American Society for Information Science Technology},
%volume    = {58},
%number    = {7},
%year      = {2007},
%pages     = {1019–1031},
%}
%
%@book{Han-2012,
  %author    = {Jiawei Han and Micheline Kamber and Jian Pei},
  %title     = {Data Mining Concepts and Techniques},
  %year      = {2012},
  %edition   = {3rd},
  %publisher = {Morgan Kaufmann}
%}
%
%@inproceedings{Agrawal-1994,
  %author    = {Rakesh Agrawal and Ramakrishnan Srikant},
  %title     = {Fast algorithms for mining association rules},
  %booktitle = {Proceedings of the International Conference on Very Large Databases},
  %year      = {1994},
  %pages     = {487-499},
  %publisher = {ACM Press}
%}
%
%

\bibliographystyle{plain}
\bibliography{myRef} %change to your file name (with suffix .bib)


\end{document} 
